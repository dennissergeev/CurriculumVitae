\documentclass[a4paper,10pt]{article}

%
% Useful Packages
%
\usepackage[english]{babel}
\usepackage{datetime}
\usepackage{marvosym}
\usepackage{MnSymbol}
\let\mathdollar\relax
\usepackage{fontspec} 					%for loading fonts
\usepackage{xunicode,xltxtra,url,parskip} 	%other packages for formatting
\RequirePackage{color,graphicx}
\usepackage[usenames,dvipsnames]{xcolor}
%\usepackage[big]{layaureo} 				%better formatting of the A4 page
% an alternative to Layaureo can be ** \usepackage{fullpage} **
%\usepackage{supertabular} 				%for Grades
\usepackage{titlesec}					%custom \section

\usepackage{tabu}

%
% Links
%
\usepackage{hyperref}
\definecolor{linkcolour}{rgb}{0,0.1,0.3}
\hypersetup{colorlinks,breaklinks,urlcolor=linkcolour, linkcolor=linkcolour}

%
% Fonts
%
%\defaultfontfeatures{Mapping=tex-text}
%%\setmainfont[SmallCapsFont = Fontin SmallCaps]{Fontin}
%%%% modified for Karol Kozioł for ShareLaTeX use
%\setmainfont[
%SmallCapsFont = Fontin-SmallCaps.otf,
%BoldFont = Fontin-Bold.otf,
%ItalicFont = Fontin-Italic.otf
%]
%{Fontin.otf}
%%%%

% 
% Column size for dates
%
\newlength{\mycol}
\setlength{\mycol}{2.3cm} 

%
% Bibliography list
%
\makeatletter
\newlength{\bibhang}
\setlength{\bibhang}{1em}
\newlength{\bibsep}
 {\@listi \global\bibsep\itemsep \global\advance\bibsep by\parsep}
\newenvironment{bibsection}%
        {\begin{enumerate}{}{%
       \setlength{\leftmargin}{\bibhang}%
       \setlength{\itemindent}{-\leftmargin}%
       \setlength{\itemsep}{\bibsep}%
       \setlength{\parsep}{\z@}%
        \setlength{\partopsep}{0pt}%
        \setlength{\topsep}{0pt}}}
        {\end{enumerate}\vspace{-.6\baselineskip}}
\makeatother

%
% Sections 
% inspired by: http://stefano.italians.nl/archives/26
%
\titleformat{\section}{\Large\scshape\raggedright}{}{0em}{}[\titlerule]
\titlespacing{\section}{0pt}{3pt}{3pt}

%--------------------BEGIN DOCUMENT----------------------
\begin{document}

\pagestyle{empty} % non-numbered pages
\flushright{\small \texttt{Updated: \shortmonthname[\the\month] \the\year}}\flushleft

%--------------------TITLE-------------
\par{\centering
		{\Huge Denis E. \textsc{Sergeev} \par
		 \normalsize \textit{Curriculum vitae}
		
	}\bigskip\par}

%--------------------SECTIONS-----------------------------------
\section{Personal Data}
\renewcommand{\arraystretch}{1.2}
\begin{tabu} to \textwidth {p{\mycol}|X[l]}
    \textsc{Address:}   & 3.16 School of Environmental Sciences, University of East Anglia, Norwich,  NR47TJ, UK \\
    \textsc{Phone:}     & +44 7518298358\\
    \textsc{email:}     & \href{mailto:d.sergeev@uea.ac.uk}{d.sergeev@uea.ac.uk}
\end{tabu}
\renewcommand{\arraystretch}{1}

 \renewcommand{\labelitemi}{\scriptsize$\blacksquare$} 

\section{Education}
\begin{tabu} to \textwidth {p{\mycol}|X[l]}
 \textsc{2014--\small{Present}} & PhD in \textsc{Meteorology}\\
& \textbf{School of Environmental Sciences} \\
& \textbf{University of East Anglia, UK} \\
& Thesis title: ``Dynamics and predictability of polar lows'' \\
& Supervisor: \href{mailto:I.Renfrew@uea.ac.uk}{Ian A. Renfrew} \\
& \quad tel.: +44 (0)1603 59 2557 \\
& \quad e-mail: \href{mailto:i.renfrew@uea.ac.uk}{i.renfrew@uea.ac.uk} \\
& \quad address: School of Environmental Sciences, University of East Anglia, Norwich NR47TJ, UK \\
\multicolumn{2}{c}{} \\
 \textsc{2009--2014} & Specialist Diploma in \textsc{Meteorology}\\
& \textbf{Faculty of Geography} \\
& \textbf{Lomonosov Moscow State University, Russia} \\
& With Honours \hfill | Average grade: 4.96/5 
\\ %\hyperlink{grds}{\hfill | \footnotesize Detailed List of Exams} \\ %\small\emph{magna cum laude}
& Thesis title: ``Idealised numerical modelling of polar mesocyclones dynamics'' \\
& Supervisor: \href{mailto:stepanen@srcc.msu.ru}{Victor M. Stepanenko}
\end{tabu}

\section{Research Experience}
\begin{tabu} to \textwidth {p{\mycol}|X[l]}
%\begin{tabular}{p{\mycol}|l}	
 {\small Oct.} 2013 & \textbf{Visiting student} \\
& Geophysical Institute, \\
& University of Bergen, \\
& Bergen, Norway \\
& Supervisor: \href{mailto:thomas.spengler@gfi.uib.no}{Thomas Spengler}\\
\multicolumn{2}{c}{} \\
{\small Jul.} 2012 & \textbf{Research Assistant} \\
& Laboratory of climate theory, \\
& A.M. Obukhov Institute of Atmospheric Physics, \\
& Russian Academy of Sciences \\
& Moscow, Russia \\
& Supervisor: \href{mailto:eliseev@ifaran.ru}{Alexey V. Eliseev}\\
\end{tabu}

\section{Reviewed Publications}
\begin{bibsection}
    \item Eliseev AV, {\bf Sergeev DE}. 2014. Impact of Subgrid Scale Vegetation Heterogeneity on the Simulation of Carbon Cycle Characteristics. \emph{Izvestiya, Atmospheric and Oceanic Physics}. \textbf{50(3):} 259--270.
\end{bibsection}

      
\section{Scholarships and Awards}
\begin{tabu} to \textwidth {p{\mycol}|X[l]}	
 \textsc{2014-2017} & Lord Zuckerman scholarship
                    \newline School of Environmental Sciences
                    \newline University of East Anglia \\
 \multicolumn{2}{c}{} \\
 \textsc{2014} & Young Scientist's Travel Award (YSTA)
                 \newline European Geosciences Union (EGU) General Assembly \\
 \multicolumn{2}{c}{} \\
 \textsc{2014} & Russian Academy of Sciences Young Scientist Medal
                 \newline In the area of oceanology, atmospheric physics and geography
\end{tabu}

\section{Grants}
\begin{tabu} to \textwidth {p{\mycol}|X[l]}	
 2014-2016 & \textbf{Characteristics of the mesoscale atmospheric circulations in the Arctic and their influence on the atmosphere-ocean energy exchange} \\
& Russian Foundation for Basic Research (RFBR) Grant
\end{tabu}

\section{Languages}
\begin{tabular}{rp{10cm}}
\textsc{Russian:} & Native speaker \\
\textsc{English:} & Fluent (IELTS 8.0)\\
\textsc{French:} & Reading Knowledge\\
\end{tabular}

\section{Computer skills}
\begin{tabular}{rp{10cm}}
	Operating systems & \textbf{Linux}, Unix, Windows \\
	Computer Languages & \textbf{Python}, Fortran \\
	Data analysis and visualizing & \textbf{Python}, MATLAB, NCL, VAPOR, GrADS \\
	Parallel programming in Fortran & MPI, OpenMP \\
	Version control systems & Git, Subversion \\
	Document preparation & \LaTeX, Microsoft Office \\
	Web development & HTML, CSS (basic level)
\end{tabular}

\end{document}
